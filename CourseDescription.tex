\section*{Introduction}
These are the courses that I completed during my bachelor's degree. The descriptions of these courses are sourced from  \href{https://docs.ce.sharif.edu/course/}{this webpage} of the Sharif University of Technology Computer Engineering Department.

Since our department or university does not provide an official English course description, I gathered this data from the mentioned official webpage containing formal official descriptions and translated them into English.
\newpage
\tableofcontents
\newpage


\section{Computer Workshop}
Course Code: 40108 \qquad \quad \href{https://docs.ce.sharif.edu/course/40108}{Description Link}
\qquad \quad Number of Credits: 1

This course provides first-year computer engineering students with essential knowledge and skills in general computer usage. It covers various fundamental topics, including an introduction to hardware, operating systems, common software applications, web programming, basic networking principles, and internet usage. The course is designed to be hands-on, with each session structured as a workshop where the instructor introduces concepts and guides students through practical exercises. Students will engage in hands-on activities to reinforce the material taught during each session. Before each class, students or the workshop administrator must install and prepare the software for the upcoming practical work.

\section{Introduction Programming}
Course Code: 40153 \qquad \quad \href{https://docs.ce.sharif.edu/course/40153}{Description Link}
\qquad \quad Number of Credits: 3

This course aims to familiarize students with programming in C, focusing on writing well-engineered code. This includes structured programming, proper use of comments, code modularization, and the ability to implement pseudocode into functional programs.

\section{Discrete Structures}
Course Code: 40115 \qquad \quad \href{https://docs.ce.sharif.edu/course/40115}{Description Link}
\qquad \quad Number of Credits: 3

The aim of this course is to familiarize students with the concepts, structures, and techniques of discrete mathematics that are widely used in computer science and engineering. The course aims to develop foundational skills, including understanding and constructing precise mathematical proofs, creative problem-solving, familiarity with fundamental results in logic, combinatorics, number theory, graph theory, and computation theory, as well as providing the necessary mathematical prerequisites for many other courses offered in various fields of computer engineering.

\section{Computer Technical Structure}
Course Code: 40211 \qquad \quad \href{https://docs.ce.sharif.edu/course/40211}{Description Link}
\qquad \quad Number of Credits: 2

This course introduces students to general concepts and key terms commonly used in the field of computer science and engineering while teaching the methods and skills needed for studying and writing academic and specialized texts in the field. The primary focus is on strengthening students' abilities to read and understand technical texts in computer science, though there will also be work on improving writing and listening comprehension skills. The course content is divided into three sections: core topics, optional topics, and supplementary topics. In each three-hour session (consisting of two hours of lecture and one hour of practice), two or three core topics will be covered. Optional topics will be taught if time permits, but they are typically assigned as exercises. Supplementary topics include studying recent developments in computer science and engaging in additional class activities, such as student presentations and multimedia use.

\section{Logic Design}
Course Code: 40212 \qquad \quad \href{https://docs.ce.sharif.edu/course/40212}{Description Link}
\qquad \quad Number of Credits: 3

The goal of this course is to familiarize students with logic gates as the circuits that implement logical operators and with simple integrated circuits built using a limited number of gates. Students will learn methods for analyzing and designing combinational circuits and synchronous sequential circuits. Additionally, they will understand the structure, functionality, and application of certain basic integrated circuits that form the foundation of more complex systems. This foundational knowledge will prepare them for further study of hardware components such as processors.


\section{Advanced Programming}
Course Code: 40244 \qquad \quad \href{https://docs.ce.sharif.edu/course/40244}{Description Link}
\qquad \quad Number of Credits: 3

This course covers object-oriented programming concepts using the Java programming language. It also explores Java's inherent features, programming capabilities, differences in approach compared to similar languages, concurrent programming, and software quality. Students are expected to have prior experience with a programming language and be familiar with problem-solving methods like search algorithms, sorting, mathematical operations, and similar topics. The main focus of this course is on object-oriented concepts.


\section{Fundamentals of Electric \& Electronic Circuits}
Course Code: 40124 \qquad \quad \href{https://docs.ce.sharif.edu/course/40124}{Description Link}
\qquad \quad Number of Credits: 3

This course aims to introduce students to electrical components and methods for analyzing electrical circuits both in the time domain and the Laplace domain. Additionally, students will learn about the electronic circuits that form logic gates across several widely-used technologies.

\section{Computer Structure \& Language}
Course Code: 40126 \qquad \quad \href{https://docs.ce.sharif.edu/course/40126}{Description Link}
\qquad \quad Number of Credits: 3

The primary goal of this course is to familiarize students with the various components of a computer and how they interact to execute a program’s instructions. Programming in machine language and assembly, along with converting between the two, provides students with a deeper understanding of instruction set architecture and how to efficiently utilize the machine. By the end of the course, students will be prepared to learn the design and implementation of these components in a computer architecture course.
Logic Design Laboratory


\section{Logic Design Laboratory}
Course Code: 40206 \qquad \quad \href{https://docs.ce.sharif.edu/course/40206}{Description Link}
\qquad \quad Number of Credits: 1

The goal of this course is to familiarize students with the implementation of logic circuits, including shift registers, adders, subtractors, counters, registers, and data buses. The Logic Design Lab provides a hands-on experience where students can apply the theories learned in the Logic Design course.

\section{Data Structure \& Algorithms}
Course Code: 40254 \qquad \quad \href{https://docs.ce.sharif.edu/course/40254}{Description Link}
\qquad \quad Number of Credits: 3

This course introduces students to methods of algorithm analysis, as well as both basic and some intermediate but important data structures. It also covers several fundamental algorithms, with a focus on analyzing and proving their correctness. Students are expected to already be familiar with a programming language like C++ or Java, as well as with recursive problem-solving techniques. The algorithms taught in the course are independent of any specific language and are presented according to the instructions in the reference textbook.

\section{Engineering Probability \& Statistics}
Course Code: 40181 \qquad \quad \href{https://docs.ce.sharif.edu/course/40181}{Description Link}
\qquad \quad Number of Credits: 3

This course aims to introduce students to fundamental concepts in probability theory and statistical inference, along with their applications in computer engineering, such as data modeling problems like regression. Key topics include the interpretation and axioms of probability, single and multivariable probability distributions, conditional probability and statistical independence, random variables and expectation, functions of random variables, the exponential family of distributions, the central limit theorem, the law of large numbers, and hypothesis testing.


\section{Digital System Design}
Course Code: 40223 \qquad \quad \href{https://docs.ce.sharif.edu/course/40223}{Description Link}
\qquad \quad Number of Credits: 3

The objective of this course is to familiarize students with the Verilog hardware description language (HDL), teach them how to design hardware using HDL at various levels of abstraction and introduce the internal structure of programmable circuits in a practical environment. Students will also learn how to implement hardware circuits on FPGA and CPLD devices.

\section{Design of Algorithms}
Course Code: 40354 \qquad \quad \href{https://docs.ce.sharif.edu/course/40354}{Description Link}
\qquad \quad Number of Credits: 3

The goal of this course is to familiarize students with common methods for designing efficient algorithms to solve various problems. The course will emphasize the analysis of algorithm efficiency and the proof of their correctness. Key topics from algorithm theory, such as computational complexity, flow networks, and graph algorithms, will also be covered.

\section{Mobile Programming}
Course Code: 40429 \qquad \quad \href{https://docs.ce.sharif.edu/course/40429}{Description Link}
\qquad \quad Number of Credits: 3

The purpose of this course is to introduce students to concepts and patterns in mobile programming. The course will cover programming for Android and iOS operating systems. Prior knowledge of Java is required, making the completion of the Advanced Programming course a necessary prerequisite.

\section{Digital System Design Laboratory}
Course Code: 40206 \qquad \quad \href{https://docs.ce.sharif.edu/course/40206}{Description Link}
\qquad \quad Number of Credits: 1

This laboratory course provides students with hands-on experience in the design and implementation of digital systems. The focus is on using computer-aided design (CAD) tools to develop digital systems and implementing these designs on programmable devices such as CPLDs and FPGAs. Through practical exercises, students gain valuable experience in the entire digital design process, from concept to implementation, enhancing their understanding of digital logic, hardware description languages, and programmable logic devices. The course aims to bridge theoretical knowledge with practical skills in digital system design.


\section{Numerical Computations}
Course Code: 40215 \qquad \quad \href{https://docs.ce.sharif.edu/course/40215}{Description Link}
\qquad \quad Number of Credits: 3

This course aims to familiarize undergraduate students with numerical methods for solving scientific problems across various fields of science and engineering. These methods allow for approximate solutions to problems that cannot be solved exactly using traditional mathematical techniques or are too complex for direct solutions. In some cases, exact solutions are possible but highly complicated, which introduces errors. Numerical methods provide estimates with limited error and reduced complexity. At the beginning of the course, students will learn about error concepts, followed by different numerical methods used to solve engineering problems. Another key focus will be on the use of efficient software tools to solve problems, compare numerical methods, and graphically present results for summarization. Additionally, the course will introduce students to contemporary problems that are either impossible to solve with standard mathematical approaches or involve significant complexity, where numerical methods offer efficient solutions. This practical application is a core aspect of the course.


\section{Linear Algebra}
Course Code: 40282 \qquad \quad \href{https://docs.ce.sharif.edu/course/40282}{Description Link}
\qquad \quad Number of Credits: 3

This course aims to introduce students to the fundamental theoretical concepts of linear algebra and how to implement and apply them using appropriate software tools. Understanding these concepts will enable students to analyze linear mappings and systems through matrices, operators, and related defined operations. Additionally, optimization problems, one of the common applications of linear algebra, will be explored in detail.


\section{Database Design}
Course Code: 40384 \qquad \quad \href{https://docs.ce.sharif.edu/course/40384}{Description Link}
\qquad \quad Number of Credits: 3

In this course, students will become familiar with the concepts of data semantic modeling and database design. By the end of the term, students are expected to have a thorough understanding of the topics outlined in the course syllabus.

This course introduces students to the core concepts of data modeling and database design, with a focus on both relational and non-relational database systems. Students will learn about database architectures, including centralized, client-server, and distributed systems, as well as the components of relational database management systems (RDBMS). The course covers semantic data modeling using ER and EER diagrams, essential techniques like specialization, generalization, and normalization, and the principles of relational database design. Students will also gain hands-on experience with SQL for data definition, manipulation, and query formulation, and will explore key database concepts such as transaction processing, data integrity, and security. Additionally, the course introduces NoSQL databases and provides a brief overview of data warehouses and OLAP systems. Throughout the course, SQL is integrated into practical examples to reinforce learning.

\section{Compiler Design}
Course Code: 40414 \qquad \quad \href{https://docs.ce.sharif.edu/course/40414}{Description Link}
\qquad \quad Number of Credits: 3

Compiler design and construction is one of the fundamental concepts in computer science. Although there is limited diversity in compiler construction methods, they can be applied to create interpreters and translators for a wide range of languages and machines. In this course, compiler construction is introduced by describing the main components of a compiler, their tasks, and how they interact. After an introduction to the components of a compiler and different types of grammar, the various stages of translation, including lexical, syntactic, and semantic analysis, as well as code generation and optimization, will be explained.


\section{Artificial Intelligence}
Course Code: 40417 \qquad \quad \href{https://docs.ce.sharif.edu/course/40417}{Description Link}
\qquad \quad Number of Credits: 3

This course introduces both the theoretical and practical aspects of artificial intelligence (AI). The goal of the AI course is to present techniques for making optimal or near-optimal decisions in various problems and environments. Topics covered include search, problem-solving, knowledge representation, and inference. The course will also address search in uncertain environments, knowledge representation in such settings, and probabilistic inference for decision-making under uncertainty. Additionally, an introduction to machine learning will be provided. Finally, students will become familiar with several practical applications of AI.

\section{Computer Simulation}
Course Code: 40634 \qquad \quad \href{https://docs.ce.sharif.edu/course/40634}{Description Link}
\qquad \quad Number of Credits: 3

The purpose of this lesson is to familiarize students with simulation methods and related topics.

This course provides an introduction to various simulation methods and their applications. Students will gain practical skills using MATLAB or similar computational tools to perform simulations. The course covers fundamental concepts of simulation, including discrete-event simulation and system modeling. Key topics include statistical models used in simulation, generating random numbers, and methods for random variable generation such as the inverse transform and acceptance-rejection methods. Students will learn about data collection, distribution fitting, parameter estimation, and model validation. The course also explores advanced simulation techniques like Monte Carlo simulations, transient and steady-state analysis of stochastic processes, experimental design, and sensitivity analysis. Real-world examples will be used to illustrate these concepts and techniques.


\section{Technical Presentation}
Course Code: 40221 \qquad \quad \href{https://docs.ce.sharif.edu/course/40221}{Description Link}
\qquad \quad Number of Credits: 2

This course aims to teach students the skills, principles, etiquette, and ethics of scientific and technical presentations, as well as effective methods for delivering structured presentations (both written and oral). The course focuses on improving the quality of these presentations and familiarizing students with managing oral presentations and writing scientific or professional documentation. Emphasis will be placed on the content, structure of components, and how to access scientific resources.

\section{Computer Architecture}
Course Code: 40323 \qquad \quad \href{https://docs.ce.sharif.edu/course/40323}{Description Link}
\qquad \quad Number of Credits: 3

In this course, students become familiar with various components of a computer and how they interact to execute program instructions. The main objective of this course is to teach the design and implementation of these components, as well as different techniques for implementing various architectures for different applications.

\section{Advanced Information Retrieval}
Course Code: 40324 \qquad \quad \href{https://docs.ce.sharif.edu/course/40324}{Description Link}
\qquad \quad Number of Credits: 3

This course introduces information retrieval systems. It begins with indexing operations and the Boolean information retrieval model. The vector space model and tf-idf representation are then presented, followed by a discussion on techniques for accelerating document scoring and ranking. Probabilistic information retrieval models are introduced next, along with concepts of document classification, clustering, and learning to rank. Web search engines are then explored, examining key components such as crawlers, document graph analysis, and similar document detection. At the end of the course, students will be introduced to recommender systems and advanced information retrieval concepts.

\section{Programming Language Design}
Course Code: 40364 \qquad \quad \href{https://docs.ce.sharif.edu/course/40364}{Description Link}
\qquad \quad Number of Credits: 3

This course explores the evolution and design principles of programming languages, providing a step-by-step, experiential overview of the development and implementation of various programming language generations. Students will learn about language engineering techniques used during the design and implementation phases, with a focus on domain-specific languages and their significance. The course also covers the implementation of interpreters, particularly in the context of virtual machines, and examines the fundamental principles and issues related to programming language design. Additionally, students will explore the data structures and methods used in building programming environments, gaining insights into both theoretical and practical aspects of language design and implementation.

\section{Computer Architecture Laboratory}
Course Code: 40103 \qquad \quad \href{https://docs.ce.sharif.edu/course/40103}{Description Link}
\qquad \quad Number of Credits: 1

This course aims to familiarize students with practical methods for implementing key components of computer architecture, such as the arithmetic-logic unit, control unit, and memory. Through this hands-on approach, students will gain a realistic understanding of how to design and implement an instruction set on a target architecture, while acquiring practical experience in the process.

\section{Computer Engineering Internship}
Course Code: 40450 \qquad \quad \href{https://docs.ce.sharif.edu/course/40450}{Description Link}
\qquad \quad Number of Credits: 0

The internship is a period during which students spend a specified number of work hours at approved centers, providing them with the opportunity to align their academic knowledge with practical and operational procedures used in industrial and executive environments within the country. Additionally, through this internship experience, students can assess their abilities and readiness for entering the country's industrial settings, while also identifying and planning to address any potential weaknesses.

\section{System Analysis \& Design}
Course Code: 40418 \qquad \quad \href{https://docs.ce.sharif.edu/course/40418}{Description Link}
\qquad \quad Number of Credits: 3

Just as solving a mathematical problem requires identifying its complexities and dimensions, designing appropriate strategies, and then proceeding to solve it, building computer systems for a group of people (which may often be more complex than mathematical problems) also requires analysis and design before implementation. In this course, students will become familiar with analysis and design skills such as requirements analysis, feasibility studies, modeling, process analysis, architecture design, and UI/UX design. Alongside these skills, the course will cover management and planning skills for developing computer systems, including methodologies for system development, process automation, and project management concepts specific to these types of projects.

\section{Operating System}
Course Code: 40424 \qquad \quad \href{https://docs.ce.sharif.edu/course/40424}{Description Link}
\qquad \quad Number of Credits: 3

This course aims to introduce undergraduate students to the principles of operating systems. The course includes four individual programming assignments that familiarize students with system programming. Additionally, three group programming assignments introduce students to kernel-level programming.

\section{Natural Language Processing}
Course Code: 40677 \qquad \quad \href{https://docs.ce.sharif.edu/course/40677}{Description Link}
\qquad \quad Number of Credits: 3 
\\Level of Study: Graduate (Master)

Natural Language Processing (NLP) is one of the most important branches of language processing. The goal of this field is to create a means of interaction between humans and machines through natural human language. In this course, we aim to introduce natural language processing concepts and present fundamental methods for solving problems in this domain. We will also discuss, to some extent, current approaches to addressing challenges in NLP.

This course offers a comprehensive introduction to Natural Language Processing (NLP), focusing on methods and techniques used to enable machines to understand and interact with human language. Students will explore fundamental concepts in NLP, including text preprocessing, language modeling, and text classification. The course covers preprocessing techniques such as tokenization, normalization, and stemming, and introduces foundational methods in language modeling like n-grams and perplexity. Students will learn basic text classification and clustering methods, including logistic regression and k-means clustering. The course also delves into word representation techniques, ranging from traditional methods to advanced neural network-based approaches. Key topics include machine translation, recurrent neural networks, attention mechanisms, and parsing, with practical applications such as information extraction, summarization, and part-of-speech tagging. By the end of the course, students will have a solid understanding of both foundational and advanced NLP techniques and their real-world applications.

\section{Machine Learning}
Course Code: 40717 \qquad \quad \href{https://docs.ce.sharif.edu/course/40717}{Description Link}
\qquad \quad Number of Credits: 3
\\Level of Study: Graduate (Master)

This course introduces machine learning concepts and provides an overview of its various branches, covering important theoretical and practical aspects. Different techniques and algorithms are discussed across various subfields. In supervised learning, regression and classification problems are examined, along with methods for solving these problems and evaluating models. For classification tasks, various approaches and related algorithms are presented. The unsupervised learning section covers density estimation, unsupervised dimensionality reduction, and clustering. Finally, a brief introduction to reinforcement learning is provided.

\section{Theory of Machines \& Language}
Course Code: 40415 \qquad \quad \href{https://docs.ce.sharif.edu/course/40415}{Description Link}
\qquad \quad Number of Credits: 3

This course focuses on the theoretical aspects of computer science and engineering. Topics covered include various computational models, their computational capabilities, properties, and applications. The course also explores concepts of computability, decidability, and the Church-Turing thesis regarding algorithms.


This course delves into the theoretical foundations of computer science and engineering, focusing on various computational models, their capabilities, properties, and applications. Key topics include propositional and predicate logic, proof systems, set theory, and the Russell paradox. The course covers finite automata, both deterministic and nondeterministic, and explores regular languages, expressions, and grammars. Students will study context-free languages and grammars, including parsing techniques, ambiguity, and simplification methods. The course also addresses Turing machines, the Church-Turing thesis, decidability, and the limits of computability. Additional topics include computational complexity, P and NP complexity classes, and NP-complete and NP-hard problems. Through these topics, students will gain a deep understanding of the theoretical underpinnings of computational processes and algorithms.

\section{Computer Networks}
Course Code: 40443 \qquad \quad \href{https://docs.ce.sharif.edu/course/40443}{Description Link}
\qquad \quad Number of Credits: 3

The purpose of this course is to familiarize students with the basic concepts of computer networks and related concepts.


This course provides an introduction to fundamental concepts and principles of computer networking. It covers socket programming, IP packet switching, and IP addressing and routing. Students will explore transport protocols such as TCP and UDP, congestion control mechanisms, and address translation techniques including DNS, DHCP, and ARP. The course includes topics on middleware, switches, bridges, and network links, as well as various routing methods such as connection-oriented routing, distance vector routing, and policy-based routing (BGP). Additionally, students will learn about layer-2 and peer-to-peer networks, multimedia streaming, circuit switching, wireless and mobile networks, content delivery networks (CDNs), and software-defined networking. By the end of the course, students will have a comprehensive understanding of network architecture, operations, and emerging trends in network technology.

\section{Embedded Systems}
Course Code: 40462 \qquad \quad \href{https://docs.ce.sharif.edu/course/40462}{Description Link}
\qquad \quad Number of Credits: 3

An embedded system is a computer system integrated within another system, typically non-computer-based, responsible for tasks such as managing and controlling the larger system. Statistics on computer usage show that the majority of computers worldwide (over 80 percent) are embedded systems. Furthermore, embedded systems form the foundation for important concepts in computer engineering like cyber-physical systems and the Internet of Things. This course aims to introduce the design and analysis of embedded systems. Students will also become familiar with the architecture, hardware structure, software, and programming techniques for embedded systems.


This course focuses on embedded systems, which are computer systems embedded within non-computer devices to manage and control larger systems. Emphasizing the significance of embedded systems, the course explores their applications, which account for over 80\% of all computers worldwide. Students will learn about the architecture, hardware, software, and programming aspects of embedded systems. The curriculum includes an introduction to embedded systems, microcontrollers, resource and task management, and programming. It covers popular hardware platforms like Raspberry Pi, and introduces programming based on automata and StateCharts. Additionally, the course addresses real-time constraints, energy consumption, and reliability, as well as the role of embedded systems in the Internet of Things (IoT).

\section{Software Engineering}
Course Code: 40474 \qquad \quad \href{https://docs.ce.sharif.edu/course/40474}{Description Link}
\qquad \quad Number of Credits: 3

This course aims to address important engineering principles that should be observed throughout all stages of software development. Students have already been introduced to software construction (programming), requirements analysis, and software design in previous courses. The goal here is not to teach a new method for requirements analysis or software design, but rather to teach software production as an engineered product, similar to products created in other engineering disciplines. The course begins by highlighting the differences between products developed using engineering methods and those created through artistic approaches. It then explains the expectations that an engineered product should meet. The course continues by emphasizing engineering production methods, including modeling, measurability and evaluation, verification and validation of intermediate products, and reviews scientific achievements in this field across all stages of software development. Given that students have had less exposure to formal requirements specification, measurement, estimation, and testing in previous courses, these chapters receive greater emphasis in this course. Finally, supporting activities such as project management, scheduling, risk management, configuration management, and quality assurance are reviewed, focusing on their impact on producing software in an engineered manner.

\section{Computer Engineering Project}
Course Code: 40760 \qquad \quad \href{https://docs.ce.sharif.edu/course/40760}{Description Link}
\qquad \quad Number of Credits: 3

The objective of the undergraduate project is to analyze, design, and implement a real-world project or conduct a research project based on the knowledge acquired throughout the bachelor's degree program.

\section{Software Engineering Laboratory}
Course Code: 40404 \qquad \quad \href{https://docs.ce.sharif.edu/course/40404}{Description Link}
\qquad \quad Number of Credits: 1

This course is offered to undergraduate students and aims to address practical aspects of software engineering. In this course, engineering methods are experienced through hands-on projects. The laboratory work covers five main areas of software engineering: requirements engineering, analysis, design, implementation, and testing. The lab consists of 10 three-hour sessions. Each group is assigned a project on which they will work throughout the semester, developing both the project itself and its documentation.

\section{Operating System Laboratory}
Course Code: 40408 \qquad \quad \href{https://docs.ce.sharif.edu/course/40408}{Description Link}
\qquad \quad Number of Credits: 1

The purpose of this laboratory is to teach various components of the Linux operating system, their usage, and the implementation of algorithms within each component. After completing this lab, students will be familiar with the structure of the Linux operating system and will have the ability to modify and compile it. The general outline of this lab is as follows, but the details of each experiment may vary from one semester to another. During a semester, not all sections of the syllabus may be covered, but the initial experiments will be covered in all semesters, followed by a focus on different topics.


\section{Computer Networks Laboratory}
Course Code: 40416 \qquad \quad \href{https://docs.ce.sharif.edu/course/40416}{Description Link}
\qquad \quad Number of Credits: 1

The Computer Networks Laboratory, offered to undergraduate students, serves as a complementary course to Computer Networks. In this course, students gain hands-on experience with some of the important concepts they've learned in the Computer Networks class. The laboratory is conducted over ten three-hour sessions.

\section{Data \& Networks Security}
Course Code: 40441 \qquad \quad \href{https://docs.ce.sharif.edu/course/40441}{Description Link}
\qquad \quad Number of Credits: 3

The purpose of this course is to introduce students to the basic concepts of security, defensive strategies, and attacks in the domains of system, web, network, and mobile security.

This course introduces fundamental concepts in data and network security, focusing on defensive strategies and attack methods across systems, web applications, networks, and mobile environments. It covers essential topics such as security policies, access control models, and hidden channels. Students will explore system security, including software execution vulnerabilities, attack techniques (e.g., control hijacking), and secure coding practices (static and dynamic analysis). The course also addresses web security concerns like SQL injection, XSS, and CSRF, as well as session management and cryptographic methods (symmetric, asymmetric, hash functions). Network security topics include protocol threats, defensive tools (IDS, VPN, firewalls), and denial-of-service attacks.

\section{General Math 1}
Course Code: 22015 \qquad \quad \href{https://docs.ce.sharif.edu/course/22015}{Description Link}
\qquad \quad Number of Credits: 4

This course covers fundamental concepts in mathematical analysis, starting with the properties of numbers and progressing to sequences and numerical series. It introduces the concepts of limits, continuity, and differentiation of functions, with an emphasis on transcendental functions. The course also explores the applications of derivatives in various contexts. Students learn about integrals, their applications, and advanced integration techniques. Finally, the course introduces power series and their convergence, equipping students with essential tools for mathematical problem-solving and analysis.

\section{General Math 2}
Course Code: 22015 \qquad \quad \href{https://docs.ce.sharif.edu/course/22015}{Description Link}
\qquad \quad Number of Credits: 4

This course extends the foundational principles of calculus to multivariable functions and vector calculus. It begins with an introduction to vector functions and the geometry of three-dimensional space, although students are encouraged to review this material independently. The course covers vector-valued functions, their derivatives, the concept of curves, and curve parameterization based on arc length. Multivariable calculus is introduced through partial derivatives, higher-order derivatives, the chain rule, gradients, and directional derivatives, as well as the implicit function theorem. Applications of partial derivatives, including critical points and optimization using Lagrange multipliers, are explored. The course also covers double and triple integrals, including their evaluation in polar and other coordinate systems, as well as vector fields and line integrals. Students will study surface integrals, vector field theory, and key theorems such as Green’s theorem, the divergence theorem, and Stokes' theorem, which are essential for understanding advanced topics in physics and engineering.


\section{Differential Equations}
Course Code: 22034 \qquad \quad \href{https://docs.ce.sharif.edu/course/22034}{Description Link}
\qquad \quad Number of Credits: 3

This course focuses on introducing ordinary differential equations (ODEs) and their applications in mathematical modeling. Students will explore various types of differential equations, including exact equations, linear ODEs with constant coefficients, and select nonlinear equations. The course covers homogeneous and non-homogeneous differential equations and their solution methods. Additionally, students will be introduced to the Laplace transform and its use in solving differential equations. The course also includes an introduction to systems of differential equations and techniques for solving specific types of systems, providing essential tools for analyzing dynamic systems in engineering and the sciences.

\section{Other General Courses}
These courses are taught by other departments. They are not that much related to my field of study and are general courses about physics, history, religion, physical education, and family.


\begin{itemize}
\item Physics 1
\item Introduction Persian Literature
\item Foreign Language
\item Physics 2
\item Physical Education
\item Physics Laboratory 2
\item The History of Imamat
\item General Workshop
\item Quran Subject Interpretation
\item Islamic Revolution of Iran
\item Islamic Ethics
\item Sport 1
\item Islamic Thought 1
\item Islamic Thought 2
\item The Knowledge of Family and Population
\end{itemize}